\documentclass[12pt, twocolumn]{article}

\title{Ethics Paper}
\author{Liam Bowen}

\begin{document}
\title{Ethics Paper}
\author{Liam Bowen}
%\date{14 Apr, 2022}

\maketitle

A big ethical issue with the project is the creation of inappropriate usernames and team names by users. An integral part of my project is for users to be able to have a login to keep coming back to the same team and avoid having other people mess with their teams. In order to allow for this to happen, users need to create a username to login to their account each time they go on the app. An ethical issue here is when users create a name that can be deemed offensive or rude to others. While not every person will come up with a team name that will be offensive, there needs to be a system in place to prevent any inappropriate content on the app. I see it being highly possible that a user can create a name that includes a swear word or a phrase that is offensive to a specific group of people. A name with inappropriate words written in it can offend others and create a hostile environment on the app. This is a problem that will result in a loss of some of the target audience. The creation of a hostile environment will create a space that a lot of potential players will feel scared to join because of the vulgar language used by their opponents. 

A game that has experienced problems with something similar to this is the Call of Duty franchise. Just under a year ago, the technology teams of multiple Call of Duty games announced that they banned a collective 350,000 accounts because they included racist and vulgar language in their names \cite{Staff}. Given the opportunity to have freedom in the creation of their online username, there are a lot of people that enjoy creating a vulgar name. It is possible to create a safeguard when they are first creating an account that searches for keywords, but people are creative and always can find special ways to write rude and offensive words in their usernames. From capital letters to numbers to special characters and more, there are plenty of ways that people can create an offensive name that a program doesn’t recognize, but other users will. 

There is an issue of transparency with my project. I am arguing that the problem with fantasy baseball is that it is boring because it is too simple. However, one issue I see here is that people may struggle to understand my argument. I want to make fantasy baseball more complicated to try and keep people playing the game further into the season. The issue that can be argued is that fantasy baseball won't get more interesting by complicating it. People will have trouble understanding that my goal is to put a positive spin on a game that has become boring and stagnant for the fans of the game. What people might have difficulty understanding is that the purpose of the changes I will be implementing is to create an environment that is fun for players for the duration of the whole season. The issue there is that more devoted fans to baseball that already can compete throughout an entire season will struggle to comprehend my ideas. They will take issue with the fact that I am trying to take a game that they enjoy the simplicity of and overcomplicate it. These devoted fans might not be open to a change rather than confused by the change. 

The target audience of my project is people who play fantasy baseball who feel that they need a new aspect to the game in order to keep it engaging. My changes are supposed to create a twist that will cause players to develop a new strategy for success. In an article on Bleacher Report, Caleb Garling argues that changes to fantasy baseball can make the game more strategic. \cite{Garling}. However, he mentions that the changes must stay within a reasonable level in order to make sure that it is a manageable amount of strategy for people. My proposed changes together could be too much and will implement too much strategy into the game and cause problems. Fantasy baseball will become too chaotic for players and do exactly the opposite of my goal. Instead of attracting people to play the game for longer, too much strategy will make the game too difficult and hard to follow. The scope of several modifications all together in the project might be too much and create a format that annoys players. 

There is a major ethical concern stemming from the privacy and security of people's personal information. In order to have an account on the web app, each user needs to use a secure password. Since passwords are typically hard to remember and many people do not trust storing their passwords in a password manager, people will more often than not enter a password that they use for other personal accounts. I have to store each person's passcode with their username to allow them to log back in the next time they wish to use the app. The problem with that revolves around me sharing my code with other people. If I share my code online for other people to experiment with, any passwords stored in my code will be available for others to see.

This will be a problem depending on how I approach storing the passwords. One forum post discusses a company using clear text as a place to store user information \cite{Yuck}. The original post to the forum is worried about keeping user information safe. They see a big problem potentially stemming from people who use a common password across accounts because if someone were to get hold of the information, they could further access private data about people whose passwords were leaked. They mention the potential risk of a disgruntled employee leaving the company, but not before distributing the information of others. The structure of my project is different since I won't have a team of workers, and therefore no risk of any employees distributing private information. However, the problem is still relevant. As I mentioned before, the major issue could come from me sharing my work with other people. Not only is it considered an ethical issue, but it also becomes a bit of a legal issue to distribute the personal data of others. 

Another concern for people's privacy is hackers \cite{Yuck}. If my app were to get hacked, people's private information is going to be in the wrong hands. The hacker can use that personal data for their own benefit, either by using it to reach into people's accounts themselves, or by selling the information to a third party. Then, people's information is long gone and there is hardly any way of tracking who may have it and how many people's information they may have.

\begin{thebibliography}{4}

\bibitem{Flowers} Flowers, Ray. “What Is Wrong with Fantasy Baseball?” \emph{Fantasy Guru}, 3 Feb. 2020, https://www.fantasyguru.com/what-is-wrong-with-fantasy-baseball/. 

\bibitem{Garling} Garling, Caleb. “Fantasy Baseball: Can It Be More Awesome? of Course. Some Ideas for Change.” \emph{Bleacher Report}, Bleacher Report, 3 Oct. 2017, https://bleacherreport.com/articles/584066-fantasy-baseball-can-it-be-more-awesome-of-course-some-ideas. 

\bibitem{Yuck} Yuck. “Legal or Ethical Responsibility to Safely Store Passwords.” \emph{Legal or Ethical Responsibility to Safely Store Passwords - Privacy, Security}, http://www.brightjourney.com/q/legal-ethical-responsibility-safely-store-passwords. 

\bibitem{Staff} Staff, Call of Duty. “An Update, Call of Duty ANTI-TOXICITY Progress Report.” Call of Duty®, Activision Publishing, https://www.callofduty.com/content/atvi/callofduty/blog/web/en/home/2021/05/ANTI-TOXICITY-PROGRESS-REPORT.html. 

\end{thebibliography}


\end{document}